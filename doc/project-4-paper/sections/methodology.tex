
%\section{Methodology}

\subsection{Testbed -- Hardware and Software}

To conduct our evaluation we employ a total of four hardware testbeds. These include a cluster consisting of three Raspberry Pi 4 attached to an isolated wired Ethernet network via a single Gigabit Ethernet switch and a second cluster consisting of three Raspberry Pi 5 in the same hardware layout. There are two key differences relevant to timekeeping between the Raspberry-Pi 4 and the Raspberry-Pi 5: The Raspberry-Pi 5 has support for PTP hardware timestamping on the network interface, and it has an integrated real-time clock (RTC) that can be powered by an external battery, both of which the Raspberry-Pi 4 lacks~\cite{raspberry-pi-datasheets}. Furthermore, we utilize a cluster of four Xilinx ZUBoard 1CGs, which feature a combination of ARM Cortex A53 and ARM Cortex R5F cores. These boards were adapted to run Debian with the standard Xilinx Kernel for access to the larger package repository. The kernel was also patched for R8152/R8153 support for our secondary Ethernet adapters\todo{Work on descriptions}. This selection of hardware is representative of the range of embedded systems/edge devices available, which may or may not include these hardware features that we expect to provide benefits to the synchronization quality. All clusters are controlled by programmable power delivery units so that we can simulate hardware failures in each component individually.

We have selected four vendors for behavioral analysis: PTPd, LinuxPTP, SPTP and Chrony. PTPd is a traditional implementation of the PTP protocol, that despite being less maintained in recent years~\cite{ptpd-maintainers} and lacking modern features such as hardware timestamping~\cite{ptpd-manpage}, it spawned a variety of forks and is still being deployed according to the Debian package tracker~\cite{debian-popularity-contest}, perhaps due to its relative simplicity and wider support of non-Linux UNIX systems. LinuxPTP is the most deployed~\cite{debian-popularity-contest} open-source solution on Debian, with the stated goals of providing robustness while taking advantage of the kernel features provided by Linux~\cite{linuxptp-homepage}. SPTP, a very young time-synchronization protocol, server, and client, developed by Meta due to difficulties they encountered with deploying PTP, claims to offer a comparable level of time-synchronization performance while offering lower resource consumption and better resilience~\cite{facebook-sptp}. Finally, Chrony is included as a reference representing the state of the art implementation of the Network Time Protocol (NTP)\todo{cite}. It is by far the most feature-complete time-synchronization implementation of the ones we evaluated and is far more widely deployed than PTP~\cite{debian-popularity-contest}, likely due to its applicability to wide area networks. It serves as a baseline for the other implementations.
