
\section{Methodology}

\subsection{Testbed -- Hardware and Software}

\subsection{PTP Lifecycle}

\begin{figure*}
    \begin{tikzpicture}[
        start chain=stages going right,
        stage/.style={
            draw, on chain=stages,
            every on chain/.style={join}, every join/.style={-Stealth}
        },
        text depth=0cm,
        ]
        \node[stage] {Discovery};
        \node[stage] {Best Master Clock Algorithm};
        \node[stage] (calibrate) {Calibration};
        \node[stage] (step) {Clock Step};
        \node[stage] (slew) {Clock Slew};
        \node[stage] {Maintain};

        \draw[Bar-Bar] ([yshift=-0.25cm]step.south west) -- ([yshift=-0.25cm]slew.south east) node[midway, below] {Convergence};
    \end{tikzpicture}
    \caption{Different stages in the PTP lifecycle that a PTP client traverses while synchronizing its clock. }
\end{figure*}

PTP clients traverse multiple stages in a lifecycle to synchronize their clock.