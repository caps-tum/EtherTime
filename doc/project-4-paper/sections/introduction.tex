
\section{Introduction}

``Time is of the essence'', to many readers this might only be an idiom but the relevance of time is so ubiquitous that we frequently fail to appreciate its significance. While the origin of the phrase is said to be in legal contracts [cite], the necessity of time and having a common notion of time permeates all sectors, from logistics and manufacturing in industry, to the day-to-day work-life routine in private life, and even to academia, as everybody perpetually works towards the next deadline. In computer systems and communications, time has been with us from the very beginnings, with its significance showing from the most basic synchronized digital circuits even to places where one might not expect, such as in the world's most pervasive digital cryptography deployment, the SSL/TLS PKI.

Nowadays, with the availability of the internet, satellite communications and digital clocks, we often take for granted that we can tell the time anywhere and anytime. And for most use cases, a rough estimate of the current time on the order of magnitude of seconds is perfectly sufficient.
After all, sub-second granularity is irrelevant as people follow their daily schedules, and for project plans that span years or even decades a day or two more will not make a difference. \todo{colloquial/irrelevant}
However, in communications, real-time systems and circuitry, we often operate on an entirely different scale, with sub-nanosecond-level differences quickly becoming significant in areas like chip design [cite]. Clearly, there is a very broad range of accuracies that we need to keep time in, and it is our mission to evaluate to what degree we can rely on time on modern computer systems in a range of contexts.