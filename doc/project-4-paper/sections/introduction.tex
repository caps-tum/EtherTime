
\section{Introduction}
\todo{Find places to insert the word ``real-time'' to solidify applicability to RTSS.}
%``Time is of the essence'', to many readers this might only be an idiom but the relevance of time throughout our daily lives, processes and systems is so ubiquitous that we frequently fail to appreciate its significance. While the phrase is said to have originated from the legal domain, time-criticality and the necessity of having a common notion of time permeates all sectors, from logistics and manufacturing in industry, to service-level objectives in service providers, emergency response in public healthcare systems, activities in people's day-to-day routines, and even to academia, as everybody perpetually works towards the next deadline. In computer systems and communications, time has been with us from the very beginnings, with its significance showing from the most basic synchronized digital circuits even to places where one might not expect, such as in the world's most pervasive digital cryptography deployment, the SSL/TLS PKI~\cite{ssl-client-warnings}.

Time is both ubiquitous and critical, no computer system today can run without it. With the availability of the internet, satellite communications and digital clocks, we take for granted that we can tell the time anywhere and anytime.
%
%Nowadays, with the availability of the internet, satellite communications and digital clocks, we often take for granted that we can tell the time anywhere and anytime. And for most human use cases, a rough estimate of the current time on the order of magnitude of seconds is perfectly sufficient.
%After all, sub-second granularity is irrelevant as people follow their daily schedules, and for project plans that span years or even decades a day or two more will not make a difference. \todo{colloquial/irrelevant}
However, in communications, real-time systems and circuitry, we operate on a scale where things are not so simple, with sub-nanosecond-level accuracies gaining significance in areas like networks and chip design~\cite{nanopu,sub-nanosecond-comms-design}.
There is a broad range of accuracies that we need to keep time in, but to what degree we can truly rely on modern   computer timekeeping in distributed embedded systems when failures could affect correctness?
We evaluate the performance and resilience of four implementations of prevalent time synchronization protocols, including the Precision Time Protocol (PTP) and the Network Time Protocol (NTP), across two hardware testbeds, showing strengths and weaknesses in a variety of configurations. We offer the following contributions:

\begin{itemize}
    \item Development of a tool, PTP-Perf, for fair and comparable cross-vendor evaluation of multiple time synchronization protocols and implementations,
    \item Establishment of a baseline of observed time-synchronization performance on several hardware testbeds across four vendors,
    \item Evaluation of synchronization resilience against adverse conditions and possibilities of mitigating risk of synchronization failure,
    \item Provide lessons-learned and best practices for configuring time-synchronization deployments for high reliability and availability.
\end{itemize}

The rest of this paper is structured into the following sections: Section~\ref{sec:motivation} motivates typical use-cases of time synchronization, \todo{Sections once structure is final}.