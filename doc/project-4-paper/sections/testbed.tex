\section{Testbed}\label{sec:testbed}
%\subsection{Testbed -- Hardware and Software}


\begin{table}
    \caption{Embedded Platforms Evaluated (RTC is Real-Time Clock)}
    \begin{tabular}{lllll}
    \toprule
               \textbf{Type} & \textbf{\#} & \textbf{H/W Timestamp} &    \textbf{RTC}     & \textbf{Kernel}\\
    \midrule
        Raspberry-Pi 4 &     3 &   $\times$   &  $\times$  & 6.6.20+rpt-rpi-v8 \\
        Raspberry-Pi 5 &     3 &  \checkmark  & \checkmark & 6.6.20+rpt-rpi-2712 \\
                Xilinx &     4 &  \checkmark  &            & Petalinux 6.6.10 \\
           Jetson TK-1 &     2 &  $\times$    &		       & Tegra 3.10.40\\
    \bottomrule
    \end{tabular}
    \label{tbl:hardware_testbeds}
\end{table}


We employ a hardware testbed with four different types of embedded platforms
based on two generations of Raspberry-Pis~\cite{raspberry-pi-4,raspberry-pi-5},
Xilinx AVNet~\cite{avnet-zug},
and NVIDIA Jetson TK-1 boards~\cite{nvidia-jetson-tk1}.
These are summarized in \cref{tbl:hardware_testbeds}.
All platforms run Debian 12 and Ubuntu 22.04.

%We use three Raspberry Pi 4 and three Raspberry Pi 5 boards.
Unlike the Raspberry-Pi 4,
Raspberry Pi 5 has support for hardware
timestamping (see \cref{sec:ptp-overview} for details) on the network interface
and an integrated real-time clock (RTC) that can be powered by an
external battery~\cite{raspberry-pi-datasheets}.

%Furthermore,
We utilize a
cluster of four Xilinx ZUBoard 1CGs, which feature a combination of ARM Cortex
A53 and ARM Cortex R5F cores.
These boards were adapted to run Debian with the standard Xilinx Kernels for
access to the same software packages.
The base kernel was 5.15, which we patched with R8152/R8153 drivers for
secondary Ethernet adapter support.
However, we observed that under Kernel 5.15 hardware
timestamping only worked on the transmission path for this board but not on the
receiving path, which degraded synchronization accuracy and prevented SPTP from
running at all, so we replaced it with a Xilinx 6.6.10 kernel.

Finally, we have NVIDIA Jetson TK-1 boards on 32-bit ARMv7,
painstakingly updated to Ubuntu 22.04 LTS for software compability but still
running the comparatively ancient NVIDIA-customized 3.10.40 kernel.

Our selection of hardware is representative
of a range of embedded systems available between 2014 and 2024.
They include both 32- and 64-bit platforms,
both with and without real-time clocks and hardware timestamping,
across a range of network interface manufacturers.

\change{All embedded platforms are connected to our distributed testbed \toolName{},
as shown in \cref{fig:testbed}.}
%\todo{\toolName{} description}
%and the testbed is illustrated in \cref{fig:testbed}.
%All clusters are controlled by programmable power delivery units so that we can simulate
%hardware failures in each component individually.

%To conduct our evaluation we employ a total of four hardware testbeds (\cref{tbl:hardware_testbeds,fig:testbed}) running Debian 12 and Ubuntu 22.04.
%%These include a cluster consisting of three Raspberry Pi 4 attached to an isolated wired Ethernet network via a single Gigabit Ethernet switch and a second cluster consisting of three Raspberry Pi 5 in the same hardware layout.
%There are two key differences relevant to timekeeping between the Raspberry-Pi 4 and the Raspberry-Pi 5: The Raspberry-Pi 5 has support for PTP hardware timestamping on the network interface, and it has an integrated real-time clock (RTC) that can be powered by an external battery, both of which the Raspberry-Pi 4 lacks~\cite{raspberry-pi-datasheets}. Furthermore, we utilize a cluster of four Xilinx ZUBoard 1CGs, which feature a combination of ARM Cortex A53 and ARM Cortex R5F cores. These boards were adapted to run Debian with the standard Xilinx Kernels for access to the same software packages. The base kernel was 5.15, which we patched with R8152/R8153 drivers for secondary Ethernet adapter support. However, we observed that under Kernel 5.15 hardware timestamping only worked on the transmission path for this board but not on the receiving path, which degraded synchronization accuracy and prevented SPTP from running at all, so we replaced it with a Xilinx 6.6.10 kernel. Finally, we have NVIDIA Jetson TK-1 boards on 32-bit ARMv7, painstakingly updated to Ubuntu 22.04 LTS for software compability but still running the comparatively ancient NVIDIA-customized 3.10.40 kernel. This selection of hardware is representative of a range of embedded systems/edge devices available between 2014 and 2024, both 32-bit and 64-bit and with/without real-time clocks and hardware timestamping across a range of network interface manufacturers. All clusters are controlled by programmable power delivery units so that we can simulate hardware failures in each component individually.

% # ChatGPT Latex skillz
\newcommand{\mergechars}[2]{%
  {\ooalign{#1\cr\hidewidth#2\hidewidth\cr}}%
}
\newcommand{\boardlabel}[1]{\mergechars{\large\faMicrochip}{\textcolor{white}{\raisebox{0.5ex}{\textbf{\sffamily #1}}}}}
\begin{figure}
    \begin{tikzpicture}[
        board/.style={align=center, on chain=boards},
        start chain=boards going right,
        node distance=0.5cm,
        on grid,
    ]
        \small
        \node[board] (rpi-4-6) {\boardlabel{4}};
        \node[board] (rpi-4-7) {\boardlabel{4}};
        \node[board] (rpi-4-8) {\boardlabel{4}};
        \node[board] () {};
        \node[board] (rpi-5-6) {\boardlabel{5}};
        \node[board] (rpi-5-7) {\boardlabel{5}};
        \node[board] (rpi-5-8) {\boardlabel{5}};
        \node[board] () {};
        \node[board] (petalinux01) {\boardlabel{X}};
        \node[board] (petalinux02) {\boardlabel{X}};
        \node[board] (petalinux03) {\boardlabel{X}};
        \node[board] (petalinux04) {\boardlabel{X}};
        \node[board] () {};
        \node[board] (tk1-1) {\boardlabel{N}};
        \node[board] (tk1-2) {\boardlabel{N}};

        \node[] (switch1) at (1.5, 1) {\faNetworkWired{}};
        \node[] (switch2) at (5.5, 1) {\faNetworkWired{}};
        \draw (switch1) -- (switch2) node [midway, below] {Isolated PTP Network};

        \node[below left=0.5cm and 0.75cm of rpi-4-6] (psu-1) {\raisebox{1ex}{\tiny\faWifi}\faBolt};
        \node[below right=0.5cm and 0.75cm of tk1-2] (psu-2) {\faBolt\raisebox{1ex}{\tiny\faWifi}};

        \node[below right=0.75 and 0.75cm of psu-1] (psu-1-label) {Smart PDU};
        \draw[dashed] (psu-1-label) -- (psu-1);

        \node[right=1.7cm of psu-1-label] (controlnet-label) {Control Net};
        \draw[dashed] (controlnet-label) -- ++(0.0, 0.5);


        \node [] (router) at (3.5, -1.15) {\faNetworkWired\,\faWifi};

        \foreach \board in {rpi-4-6,rpi-4-7,rpi-4-8,rpi-5-6,rpi-5-7,rpi-5-8}{
            \draw (\board) -- ++(0, 0.5) -| (switch1);
            \draw ([xshift=-0.2cm]\board) -- ++(0, -0.5) -| (psu-1.east);
            \draw ([xshift=+0.2cm]\board) -- ++(0, -0.65) -| (router);
        }

        \foreach \board in {petalinux01,petalinux02,petalinux03,petalinux04,tk1-1,tk1-2}{
            \draw (\board) -- ++(0, 0.5) -| (switch2);
            \draw ([xshift=0.2cm]\board) -- ++(0, -0.5) -| (psu-2.west);
            \draw ([xshift=-0.2cm]\board) -- ++(0, -0.65) -| (router);
        }

%        \draw[Bar-Bar] (tk1-2.north east) -- (switch2.north east -| tk1-2.north east) node[midway, right] {PTP Network};

        \node[right=1.5cm of router, label=right:Orchestrator] (rpi-serv) {\faServer\,\faDatabase};
        \draw[] (rpi-serv) -- (router);
    \end{tikzpicture}
    \caption{The entire evaluation setup, consisting of 3$\times$ Raspberry Pi
    {\sffamily 4}, 3$\times$ Raspberry Pi {\sffamily 5}, 4$\times$ {\sffamily
    X}ilinx 1BoardCG, 2$\times$ {\sffamily N}vidia Jetson TK-1.
    Clock synchronization operates on an isolated network, while logs are
    exported over a control network simultaneously used for orchestration.
    Power faults are controlled by smart Power Delivery Units (PDUs).}
    \label{fig:testbed}
\end{figure}
