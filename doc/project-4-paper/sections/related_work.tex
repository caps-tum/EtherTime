
\section{Methodology \& Related Work}

Clock synchronization protocols have been proposed for many different applications including packet-switched networks~\cite{sptp, white-rabbit, time-protocol-flooding, time-protocol-low-power} and their strengths and weaknesses have been evaluated analytically~\cite{clock-synchronization-packet-switched-networks}. Due to the need for dependability, multiple studies have compared how time protocols can fail~\cite{ptp-failures}, how fault tolerance and redundancy can be offered~\cite{fault-tolerant-clock-synchronization-distributed-systems}, and how timing protocols might be attacked by malicious actors~\cite{ptp-internal-attacks, byzantine-ptp}. However, we find that there is a lack of literature collecting empirical findings across implementations and protocols, with an early study~\cite{ntp-vs-ptp} from 2006 focusing on technologies that were available at the time and more recent studies~\cite{time-enough} not incorporating fault-tolerance as a central aspect.
%
To the best of our knowledge, we provide the first empirical evaluation of multiple Ethernet-based time synchronization protocols and implementations across several embedded hardware testbeds with an explicit focus on dependability and fault-tolerance. We also make the data collection and analysis tooling available open-source\todo{anonymize} to enable future studies to generate comparable results.