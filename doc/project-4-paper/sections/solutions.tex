
\section{Time Synchronization \& PTP}

Over the years, many different protocols, algorithms and solutions to time synchronization across computer networks have been explored~\cite{ntpv4-spec, sntp-spec, linuxptp, time-protocol-flooding, time-protocol-pulsesync, white-rabbit, time-protocol-wsn, time-protocol-low-power}. In today's most widely deployed network stack for general purpose computing, IP, two protocols have established themselves as standard for time synchronization: the Network Time Protocol (NTP) for Wide Area Networks (WANs) and the Precision Time Protocol (PTP) for Local Area Networks (LANs) that require a greater degree of time synchronization precision. Both protocols are standardized and each have multiple implementations available at the time of writing.

Because we are interested in evaluating time synchronization for dependable embedded systems, we will focus on PTP and derivatives as they are designed to be deployed in controlled environments such as fault-tolerant industrial control networks. PTP has wide-spread adoption in general computing and telecommunications, and also serves as a foundation for technologies such as IEEE Time Sensitive Networking (TSN).

We surveyed available PTP implementations and found that while there are plenty of commercial products available, the number of viable freely available implementations is rather limited. We excluded OpenPTP, Timebeat, and PPSi from our evaluation due to the unmaintained state of the project (last activity $>10$ years ago, the solution has since been commercialized), the reliance on heavy-weight infrastructure making it unsuitable for embedded applications, and stability issues caused by buffer overruns (bug report has been filed), respectively. White rabbit~\cite{white-rabbit}, an open extension to PTP for ultra-precise (sub-nanosecond) timing was also reviewed, but the highly specialized hardware required makes it less suitable for embedded systems.

This left us with PTPd~\cite{ptpd-manpage}, a slightly dated implementation of the protocol that is nevertheless popular and spawned multiple derivatives, and LinuxPTP, an open-source implementation that is closely tied to the Linux kernel and aims to take advantage of kernel and hardware capabilities to improve synchronization accuracy. We also include a recent addition, SPTP, a simplified version of PTP developed by Meta that claims to offer comparable performance to PTP while reducing resource consumption~\cite{sptp}. Finally, we include Chrony, a state-of-the-art open implementation of NTP as a baseline to compare the precision implementations against.

\subsection{PTP -- Background and Architecture}

PTP operates across a network and has two types of endpoints: master nodes and slave nodes. A PTP master provides signals to the slaves (Figure \ref{fig:ptp-architecture}), which each use the synchronization signal in combination with a path delay estimate to determine an estimate of the local clock offset to the master clock, which is subsequently used to discipline the local clock to keep it synchronized with the master's clock. While the actual protocol is slightly more complex and includes additional messages for synchronization leases and state management, these two core signals/message types are sufficient in principle to synchronize the system clocks. Note that obtaining a good clock signal on the master is considered out of scope for PTP, it is assumed that the master already has a high-quality clock source for the current time -- preferably an external source like an atomic clock or a GPS signal, but the master clock can also be obtained from e.g. a different PTP domain.

\begin{figure}
    \begin{tikzpicture}[client/.style={draw}, message/.style={midway, above, sloped, inner sep=0mm}, on grid]
        \node[client] (master) {PTP Master};
        \node[client, above=of master] (clock) {\faClock[regular]};
        \draw[thick, dotted] (master) -- (clock);

        \node[client, right=1.75cm of master] (switch) {\faNetworkWired};
        \draw[-Stealth] (master) -- (switch);

        \foreach \i in {1,...,3}{
            \node[client] (slave-\i) at (3.5, \i - 2) {PTP Slave};
            \draw[-Stealth] (switch) -- (slave-\i.west);
        }

        \begin{scope}[xshift=5.5cm, yshift=2cm]
            \node (master) at (0,0) {Master};
            \node (slave) at (2,0) {Slave};

            \draw[-] (master) -- ++(0, -3.75);
            \draw[-] (slave) -- ++(0, -3.75);

            \foreach \i in {0.5, 1, 1.5}{
                \draw[-Stealth] (0, -\i) -- ++(2, -0.5) node[message] {Sync};
            }

            \draw[-Stealth] (2, -2.5) -- ++(-2, -0.5) node[message] {Delay Req.} -- ++(2, -0.5) node[message, below, inner sep=0.5mm] {Delay Resp.};
        \end{scope}
    \end{tikzpicture}
    \caption{
        A PTP master provides a synchronization signal to a number of slaves so that they can keep their local clock synchronized to the master's clock (left). The clock synchronization relies on two types of signals (right): a periodic synchronization signal to distribute the current master time and the delay request/response to estimate the propagation delay.
    }
    \label{fig:ptp-architecture}
\end{figure}

\subsection{PTP Profiles}
PTP is built to be configurable, and settings include anything from the underlying transport (unicast/multicast packet switching), the delay mechanism to use (end-to-end or peer-to-peer), message frequencies and leases, as well as rules to discipline the clock. To reduce the complexity of configuring PTP, several so-called profiles are available that provide default settings for the specific use-case, such as general-purpose or ITU telecoms. To conduct our evaluation, we examine each vendor's default profile, as this profile is a widely-applicable general-purpose profile that does not require special configuration, and is thus likely to be deployed in many different contexts.

\subsection{Synchronization Performance}
It is generally understood that network clock synchronization accuracy is a function of the signal propagation delay and its variance~\cite{managin-pdv-for-ptp}\todo{An evaluation of this using our data would be nice}. A naive approach to clock synchronization that just sends a timestamp from the master to the slave would always be off by the propagation delay, but PTP uses path delay estimation to try and compensate for the propagation delay thus increasing the accuracy. In an ideal world where there is no packet delay variation, the propagation delay could be mitigated entirely, but in reality we have multiple software and hardware components, like the kernel, network stack and network hardware, that each introduce latency variability, which in turn worsens the performance of the delay compensation. Effects, such as asymmetric latencies caused by uneven loads, that cannot easily be compensated for further reduce the synchronization accuracy. Thus, limiting the packet delay variation becomes a primary concern when tuning for precision and dependability.

\subsection{Hardware Acceleration}

\begin{figure}
    \newcommand{\timestampClock}[1][100]{\textcolor{black!#1}{\faClock[regular]}}

    \begin{tikzpicture}[
        start chain=components going below,
        start chain=components2 going below,
        component block/.style={draw, minimum height=0.75cm},
        node distance=0cm,
    ]

            \begin{scope}[name prefix=stack1-, component/.style={component block, text width=3.5cm}]
                \node[on chain=components, component] at (0, 0) (PTP) {PTP Client \hfill \rotatebox{45}{\faStamp{}} \timestampClock{} \faEnvelope[regular]};
                \node[on chain=components, component] (Kernel) {Kernel \hfill \timestampClock[60] \faEnvelope[regular]};
                \node[on chain=components, component] (IP-Stack) {IP Stack \hfill \timestampClock[40] \faEnvelope[regular]};
                \node[on chain=components, component] (Hardware Queue) {Hardware Queue \hfill \timestampClock[30] \faEnvelope[regular]};
                \node[on chain=components, component] (NIC) {NIC \faEthernet \hfill \timestampClock[20] \faEnvelope[regular]};

                \node[above=of PTP] (title) {Software PTP};
            \end{scope}

            \begin{scope}[name prefix=stack2-, component/.style={component block, text width=3.5cm}]
                \node[on chain=components2, component] at (5.5, 0) (PTP) {\faEnvelope[regular] \hfill PTP Client};
                \node[on chain=components2, component] (Kernel) {\faEnvelope[regular] \hfill Kernel};
                \node[on chain=components2, component] (IP-Stack) {\faEnvelope[regular] \hfill IP Stack };
                \node[on chain=components2, component] (Hardware Queue) {\faEnvelope[regular] \hfill Hardware Queue};
                \node[on chain=components2, component] (NIC) {\faEnvelope[regular] \faClock[regular] \rotatebox{-45}{\faStamp{}}  \hfill \faEthernet{} NIC};

                \node[above=of PTP] (title) {Hardware PTP};
            \end{scope}

            \draw[Stealth-Stealth, very thick] (stack1-NIC) -- (stack2-NIC) node[midway, draw, fill=white] (network) {\faNetworkWired};

            \node[draw, dashed, below=0.25cm of network] (network-annotation) {\faClock[regular] $\rightarrow$ \faHistory{}\,\textsuperscript{\textbf{*}}};
            \draw[dotted] (network.south west) -- (network-annotation.north west);
            \draw[dotted] (network.south east) -- (network-annotation.north east);

    \end{tikzpicture}
    \caption{
        Timestamping for PTP messages when using software timestamping (left) and hardware timestamping (right). For software timestamping, the timestamp is generated inside the PTP client and traverses many layers in both egress and ingress, causing additional path delay and delay variation. With hardware timestamping, a timestamp is only added to the message just before it is written to wire by the NIC, thus ensuring a more up-to-date timestamp is sent to the network.\\
        \textsuperscript{\textbf{*}}Network hardware (switches, routers, etc.) also introduces queuing delays. Special PTP-aware hardware can compensate for its own delays to further improve timestamping quality.
    }
    \label{fig:ptp-sw-hw}
\end{figure}

Since delay variation is a primary concern, techniques have been developed to reduce the variability. While PTP can run entirely in software, the path that packets need to traverse between two PTP clients not only includes several hardware components but also some software layers (Figure~\ref{fig:ptp-sw-hw} left). Each component along the path introduces latency and packet delay variation, deteriorating the signal quality. With the appropriate hardware support, message timestamps can instead be generated directly by the NIC driver/hardware (Figure~\ref{fig:ptp-sw-hw} right), which ensures that the timestamp is not affected by the layers above it. This increases the clock synchronization's resilience against interference from adverse conditions, such as high network-, CPU- or kernel load.

However, hardware timestamping alone does not guarantee a high level of dependability. In order to determine what can go wrong, we need to take a look at what constraints the timing system needs to observe and what might lead to potential failures.
