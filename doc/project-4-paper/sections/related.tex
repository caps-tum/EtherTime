\section{Related Work}
\label{sec:related}

%\todo{I have commented out what looked like the single related work paragraph. It seemed out of place between two PTP paragraphs, which provide the necessary background for this work. We can move the related work to end if it is going to be this short.}
Clock synchronization protocols have been proposed for many different applications including packet-switched networks~\cite{sptp, white-rabbit, time-protocol-flooding, time-protocol-low-power} and their strengths and weaknesses have been evaluated analytically~\cite{clock-synchronization-packet-switched-networks}. Due to the need for dependability, multiple studies have compared how time protocols can fail~\cite{ptp-failures}, how fault tolerance and redundancy can be offered~\cite{fault-tolerant-clock-synchronization-distributed-systems}, and how timing protocols might be attacked by malicious actors~\cite{ptp-internal-attacks, byzantine-ptp}. However, we find that there is a lack of literature collecting empirical findings across implementations and protocols, with an early study~\cite{ntp-vs-ptp} from 2006 focusing on technologies that were available at the time and more recent studies~\cite{time-enough} not incorporating fault-tolerance as a central aspect.
