
%\section{Time Synchronization, PTP \& Related Work}
\section{Protocols Surveyed} % \& Related Work}
\label{sec:background}

\todo{Check out related paper suggestions by Arpan}
%Over the years,
Several time synchronization protocols, algorithms, and solutions
%across computer networks
have been explored~\cite{ntpv4-spec, sntp-spec, linuxptp, time-protocol-flooding, time-protocol-pulsesync, white-rabbit, time-protocol-wsn, time-protocol-low-power}.
%In today's most widely deployed network stack for general purpose computing, IP,
Two protocols have established themselves as the unofficial standard:
%for time synchronization:
Network Time Protocol (NTP)~\cite{ntpv4-spec} for wide area networks %(WANs)
and Precision Time Protocol (PTP)~\cite{ptp-spec} for local area networks %(LANs)
(which require a greater degree of precision).
Both protocols have standardized specifications and multiple implementations
available at the time of writing.

Since we are interested in evaluating time synchronization for distributed
embedded systems,
we focus on PTP and derivatives,
which are designed for controlled environments
%(such as fault-tolerant industrial control networks)
and have widespread adoption across multiple domains,
%general computing and
such as industrial control networks, robotics, and telecommunications.
PTP also serves as a foundation for technologies such as
IEEE Time Sensitive Networking~\cite{??}.

\todo{I have commented out what looked like the single related work paragraph. It seemed out of place between two PTP paragraphs, which provide the necessary background for this work. We can move the related work to end if it is going to be this short.}
%Clock synchronization protocols have been proposed for many different applications including packet-switched networks~\cite{sptp, white-rabbit, time-protocol-flooding, time-protocol-low-power} and their strengths and weaknesses have been evaluated analytically~\cite{clock-synchronization-packet-switched-networks}. Due to the need for dependability, multiple studies have compared how time protocols can fail~\cite{ptp-failures}, how fault tolerance and redundancy can be offered~\cite{fault-tolerant-clock-synchronization-distributed-systems}, and how timing protocols might be attacked by malicious actors~\cite{ptp-internal-attacks, byzantine-ptp}. However, we find that there is a lack of literature collecting empirical findings across implementations and protocols, with an early study~\cite{ntp-vs-ptp} from 2006 focusing on technologies that were available at the time and more recent studies~\cite{time-enough} not incorporating fault-tolerance as a central aspect.

We surveyed available PTP implementations and found that while there are plenty of commercialized options,
the number of viable freely available implementations is rather limited.
\cref{tab:vendors} summarizes the results. % of our survey.
\todo{What do you mean by viable?}
We excluded four vendors from our evaluation:
\textbf{(i)}~OpenPTP~\cite{openptp} is currently unmaintained; its last activity was more than ten years ago, and it has since been commercialized.
\textbf{(ii)}~Timebeat~\cite{timebeat} relies on heavy-weight infrastructure
(specifically the Elasticsearch/Logstash/Kibana stack) making it unsuitable for embedded applications.
\textbf{(iii)}~PPSI~\cite{ppsi} has stability issues caused by buffer overruns;
we have filed a bug report\cite{ppsi-bug-report} in this regard.\todo{@arpan: cited as personal communication: submitted via email because no public issue tracker is available}
\textbf{(iv)}~We also reviewed White rabbit~\cite{white-rabbit},
an open extension to PTP for sub-nanosecond timing.
It requires highly specialized hardware
(vendor-specific synchronous Ethernet switches and NICs with \emph{synchronization} support),
which makes it less suitable for COTS embedded systems.

This left us with four suitable vendors: PTPd, LinuxPTP, SPTP and Chrony. %(summarized in Table~\ref{tab:vendors}).
\textbf{(a)}~PTPd~\cite{ptpd-manpage} %is a traditional implementation of the PTP protocol that, despite being less
has been less maintained in recent years~\cite{ptpd-maintainers} and lacks
modern features such as hardware timestamping;
but it has spawned a variety of derivates (also commercial) and is still being
deployed as per the Debian package tracker~\cite{debian-popularity-contest}
(perhaps due to its relative simplicity and wider support of non-Linux UNIX systems).
\textbf{(b)}~LinuxPTP~\cite{linuxptp-homepage} is the most deployed open-source
PTP solution on Debian.
Its goal is to provide robustness while tightly integrating with Linux, taking advantage of the kernel and hardware features provided to improve
synchronization accuracy.
\textbf{(c)}~SPTP~\cite{facebook-sptp} is a novel time synchronization protocol
%consisting of the PTP4U server~\cite{??} and the SPTP client,
developed at Meta due to difficulties they encountered with deploying PTP.
It claims to perform on par with PTP in terms of time synchronization,
%a comparable level of time-synchronization performance
while consumes less resources and offering higher resilience.
\textbf{(d)}~Finally, we include Chrony~\cite{??}, the state-of-the-art NTP implementation.
%as a reference.
It is by far the most feature-complete implementation of time synchronization
among the ones we evaluated.
%and is far more widely deployed than PTP~\cite{debian-popularity-contest},
%likely due to its applicability to wide area networks.
As a general-purpose protocol, %for time synchronization,
it serves as a baseline for other PTP implementations.
%it serves as a baseline for the other implementations so that we can compare PTP implementation performance to the performance of general-purpose time synchronization.


\begin{table}
    \caption{Time-Synchronization Protocols Surveyed and Evaluated (\checkmark)}
    \begin{tabular}{lll}
    \toprule
        & \textbf{Vendor (Version)} & \textbf{Remarks} \\
    \midrule
        \checkmark & PTPd (2.3.1)     & Established implementation with derivatives\\
        \checkmark & LinuxPTP (3.1.1) & Advanced capabilities but specific to Linux\\
        \checkmark & SPTP (b27bdba)   & Meta's custom time synchronization protocol\\
        \checkmark & Chrony (4.3)     & State-of-the-art NTP server/client\\
        --         & OpenPTP (r2)     & Unmaintained for more than 10 years \\
        --         & Timebeat (2.0.7) & Unsuitable for embedded systems \\
        --         & PPSI (6.1)       & Critical bug found \\
        --         & White Rabbit (-) & Requires specialized hardware \\
    \bottomrule
    \end{tabular}
    \label{tab:vendors}
\end{table}

%\begin{table}
%    \caption{Surveyed Time-Synchronization Protocols}
%    \begin{tabular}{lcc@{\,}rl}
%        Vendor & Protocol & \multicolumn{2}{c}{Included} & Notes\\\hline
%        PTPd & PTP & \checkmark & 2.3.1 & Established implementation with derivatives\\
%        LinuxPTP & PTP & \checkmark & 3.1.1 & Advanced capabilities but specific to Linux\\
%        SPTP\textsuperscript{*} & Custom & \checkmark & b27bdba & Meta's custom time protocol\\
%        Chrony & NTP & \checkmark & 4.3 & Widespread NTP server/client\\
%        OpenPTP & PTP & $\times$ & r2 & Unmaintained \\
%        Timebeat & PTP & $\times$ & 2.0.7 & Unsuitable for embedded \\
%        PPSI & PTP & $\times$ & 6.1 & Critical bug \\
%        White Rabbit & Custom & $\times$ & - & Specialized hardware requirements\\
%    \end{tabular}
%    \label{tab:vendors}
%    \textsuperscript{*} Note that we had to patch 32-bit support into SPTP for our ARMv7 boards.
%\end{table}
