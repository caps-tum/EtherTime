
% Key-system

\newcommand{\ptpperfLoadKeys}[1]{\pgfkeys{#1}}
\newcommand{\ptpKeyPrefix}{/ptpperf}
\newcommand{\ptpKey}[1]{\pgfkeysifdefined{\ptpKeyPrefix/#1}{\pgfkeysvalueof{\ptpKeyPrefix/#1}}{0}}
\newcommand{\fTime}[2][0]{\fpeval{round((#2) * 10e5, #1)}\,\textmu{}s}
\newcommand{\fTimeMS}[2][0]{\fpeval{round((#2) * 10e2, #1)}\,ms}
\newcommand{\fTimeMin}[2][0]{\fpeval{round((#2) / 60, #1)}\,min}
\newcommand{\fTimeKey}[2][0]{\fTime[#1]{\ptpKey{#2}}}
\newcommand{\fTimeMath}[2][0]{\fTime{\fpeval{round(#2, #1)}}}
\newcommand{\fNum}[2][0]{\fpeval{round(#2, #1)}}
\newcommand{\fRatio}[2][0]{\fpeval{round(#2, #1)}$\times$}
\newcommand{\fPercentage}[2][0]{\fpeval{round(100*(#2), #1)}\%}
\newcommand{\fRelative}[2][0]{\fpeval{round(100*(#2)-100, #1)}\%}
\newcommand{\fRelativeInverted}[2][0]{\fpeval{100-round(100*(#2), #1)}\%}

\definecolor{color-ptpd}{HTML}{1f77b4}
\definecolor{color-linuxptp}{HTML}{ff7f0e}
\definecolor{color-sptp}{HTML}{2ca02c}
\definecolor{color-chrony}{HTML}{d62728}
\definecolor{color-border}{HTML}{cccccc}


% Format vendor ids to names
\pgfkeys{
    /ptpperf/vendors/.cd,
    ptpd/.initial={PTPd},
    ptpd/color/.initial={color-ptpd},
    linuxptp/.initial={LinuxPTP},
    linuxptp/color/.initial={color-linuxptp},
    sptp/.initial={SPTP},
    sptp/color/.initial={color-sptp},
    chrony/.initial={Chrony},
    chrony/color/.initial={color-chrony},
    /ptpperf/clusters/.cd,
    rpi-4/.initial={Raspberry-Pi 4},
    rpi-5/.initial={Raspberry-Pi 5},
    /ptpperf/vendorcluster/.cd,
    ptpd/rpi-4/.initial={PTPd on Raspberry-Pi 4},
    linuxptp/rpi-4/.initial={LinuxPTP on Raspberry-Pi 4},
    sptp/rpi-4/.initial={SPTP on Raspberry-Pi 4},
    chrony/rpi-4/.initial={Chrony on Raspberry-Pi 4},
    ptpd/rpi-5/.initial={PTPd on Raspberry-Pi 5},
    linuxptp/rpi-5/.initial={LinuxPTP on Raspberry-Pi 5},
    sptp/rpi-5/.initial={SPTP on Raspberry-Pi 5},
    chrony/rpi-5/.initial={Chrony on Raspberry-Pi 5},
}
\newcommand{\fVendor}[1]{\pgfkeysvalueof{/ptpperf/vendors/#1}}
\newcommand{\fCluster}[1]{\pgfkeysvalueof{/ptpperf/clusters/#1}}
\newcommand{\fVendorCluster}[1]{\pgfkeysvalueof{/ptpperf/vendorcluster/#1}}
\newcommand{\vendors}{ptpd,linuxptp,sptp,chrony}

% Comparison Tools

\newcommand{\cmpMin}{}
\newcommand{\cmpMinArg}{}
\newcommand{\cmpMax}{}
\newcommand{\cmpMaxArg}{}
\newcommand{\cmpMean}{}
\newcommand{\cmpCount}{}

% Search for a value: #1: counter macro, #2: values, #3: optimization functio (min/max), #4: Function value to calculate
\newcommand{\cmpSearch}[3]{%
    \xdef\cmpMin{}%
    \xdef\cmpMinArg{}%
    \xdef\cmpMax{}%
    \xdef\cmpMaxArg{}%
    \xdef\cmpMean{}%
    \xdef\cmpCount{0}%
    % Loop across values searching for the best one
    \foreach #1 in {#2}{%
        \xdef\currentValue{\fpeval{#3}}%
        %
        % If first iteration, then define best value.
        \ifundef{\cmpMin}{\xdef\cmpMin{\currentValue}}{}%
        \ifundef{\cmpMax}{\xdef\cmpMax{\currentValue}}{}%
        %
        % Update best value and if changed, then also update best argument.
        \xdef\cmpMin{\fpeval{min(\currentValue, \cmpMin)}}%
        \strcmpfullexpand{\currentValue}{\cmpMin}{\xdef\cmpMinArg{#1}}{}%
        %
        % Same for max.
        \xdef\cmpMax{\fpeval{max(\currentValue, \cmpMax)}}%
        \strcmpfullexpand{\currentValue}{\cmpMax}{\xdef\cmpMaxArg{#1}}{}%
        %
        \xdef\cmpMean{\fpeval{\cmpMean + \currentValue}}%
        \xdef\cmpCount{\fpeval{\cmpCount + 1}}%
%        Trace: \vendor, \currentValue, \cmpValue, \cmpArg\\
    }%
    %
    \xdef\cmpMean{\fpeval{\cmpMean/\cmpCount}}
}

\newcommand{\cmpSearchVendor}[1]{\cmpSearch{\vendor}{ptpd,linuxptp,sptp,chrony}{#1}}
\newcommand{\cmpSearchVendorCluster}[1]{\cmpSearch{\vendor/\cluster}{ptpd/rpi-4,linuxptp/rpi-4,sptp/rpi-4,chrony/rpi-4,ptpd/rpi-5,linuxptp/rpi-5,sptp/rpi-5,chrony/rpi-5}{#1}}
\newcommand{\cmpSave}[1]{%
    \pgfkeys{
        \ptpKeyPrefix/cmp/.cd,
        #1/min/.initial/.expanded={\cmpMin},
        #1/minarg/.initial/.expanded={\cmpMinArg},
        #1/max/.initial/.expanded={\cmpMax},
        #1/maxarg/.initial/.expanded={\cmpMaxArg},
        #1/mean/.initial/.expanded={\cmpMean},
    }%
}
\newcommand{\cmpLoad}[1]{%
    \xdef\cmpMin{\ptpKey{cmp/#1/min}}%
    \xdef\cmpMinArg{\ptpKey{cmp/#1/minarg}}%
    \xdef\cmpMax{\ptpKey{cmp/#1/max}}%
    \xdef\cmpMaxArg{\ptpKey{cmp/#1/maxarg}}%
    \xdef\cmpMean{\ptpKey{cmp/#1/mean}}%
}
\newcommand{\cmpKey}[2]{\ptpKey{cmp/#1/#2}}
\newcommand{\cmpRatioVendorClusterVsBaseline}[1]{\ptpKey{#1/\cluster/\vendor/q50}/\ptpKey{base/\cluster/\vendor/q50}}

\newcommand{\assertError}[1]{\textbf{\textcolor{red}{ASSERTION ERROR: #1}}\errmessage{ASSERTION ERROR: #1}}
% Raise an error if #2 != #3, with optional message #1.
\newcommand{\assert}[3][]{\strcmpfullexpand{#2}{#3}{}{\assertError{\strcmpfullexpand{#1}{}{#2 != #3}{#1}}}}
\newcommand{\assertRange}[3]{\pgfmathparse{ifthenelse(and(\fpeval{#1} >= #2, \fpeval{#1} <= #3),"1","0")}\assert[\fpeval{#1} not in range (#2, #3)]{\pgfmathresult}{1}}

% Utility

\makeatletter
\newcommand*{\strcmpfullexpand}[4]{%
  \ifnum\pdf@strcmp{#1}{#2}=\z@#3\else#4\fi%
}
\makeatother

\newcommand{\legend}[1][ptpd,linuxptp,sptp,chrony]{
    \begin{tikzpicture}[start chain=legend going right]
        \foreach \vendor in {#1}{
            \node[node distance=1em, on chain=legend, draw, fill=\ptpKey{vendors/\vendor/color}] {};
            \node[node distance=0cm, on chain=legend, text depth=0cm] {\sffamily\scriptsize\fVendor{\vendor}};
        }
%        \node[fit=(legend-begin)(legend-end), inner xsep=1mm, inner ysep=0mm, draw=color-border, rounded corners=5pt] {};
    \end{tikzpicture}
}
